\begin{frame}{Introduction}
    \begin{block}{Goals}
        \begin{itemize}
            \item Compute the MUSes extraction as explained in On Explaining Random Forests with SAT paper by Yacine Izza and Joao Marques-Silva
            \item Implement a method to manipulate auxiliary variables introduced during the encoding process to make sure there is no collision between identifiers of the variables
            \item Implement a method to differentiate the majority voting encoding in the case with two classes and in the case with more classes
        \end{itemize}
    \end{block}
    \vfill
    \begin{center}
        \fontsize{7}{10}\selectfont
        \textbf{PAPER GITHUB REPOSITORY}\\
        \url{https://github.com/yizza91/RFxpl}\\
        \vspace{1em}
        \textbf{OUR GITHUB REPOSITORY}\\
        \url{https://github.com/gaggioaxel/RF_SAT_explainability}
    \end{center}
\end{frame}

\begin{frame}{Introduction}
    \begin{block}{Proposed Paper}
    %\fontsize{7}{10}\selectfont
        \begin{itemize}
            \item Brief introduction to ML Classification, Decision Trees and Random Forest Classifiers
            \item Answers the question of whether finding explanations of RFs can be solved in polynomial time negatively, by proving that deciding whether a set of literals is a PI-explanation of an RF is DP-complete
            \item Proposes a propositional encoding for computing explanations of RFs, thus enabling finding PI-explanations with a SAT solver
        \end{itemize}
    \end{block}
\end{frame}

\begin{frame}{Introduction}
    \begin{block}{Symbology}
        \begin{itemize}
            \item \(\mathcal{F}\) = \( \{ 1, ..., m \} \) is the set of features
            \item \(\mathcal{K}\) = \( \{ c_{1}, ..., c_{K} \} \) is the set of classes
            \item $\mathbf{x}$ = \( \{ x_{1}, ..., x_{m} \} \) is the set of arbitrary points in the feature space $\mathbb{F}$
            \item $\mathbf{v}$ = \( \{ v_{1}, ..., v_{m} \} \) is the set of specific points in the feature space $\mathbb{F}$
            \item A pair ($\mathbf{v}$, c) is an instance where $\mathbf{v} \in \mathbb{F}$ and c $\in \mathcal{K}$
            \item M is the number of trees of the Random Forest
            \item $\mathcal{T}_{i}$ is the \textit{i}-th tree
            \item $l_{ik}$ is the classification $c_{k}$ for the \textit{i}-th tree
            \item \(\mathcal{R}_{i}\) is the set of paths of \(\mathcal{T}_{i}\)
            \item R$_{k}$ is the $k$-th path of a set \(\mathcal{R}_{i}\)
        \end{itemize}
    \end{block}
\end{frame}