\begin{frame}{Abductive Explanations (AXps)}
    \footnotesize
    \begin{block}{Definition}
        A PI-explanation (AXp) is any minimal subset  \(\mathcal{X} \subseteq \mathcal{F} \) such that:
        \begin{equation*}
            \forall (\mathbf{x} \in \mathbb{F}), \, \left[ \bigwedge_{i \in \mathcal{X}} (x_i = v_i) \right] \rightarrow (\tau(\mathbf{x}) = c)
        \end{equation*}
    \end{block}
    \begin{block}{Soft and Hard Clauses \(\langle \mathcal{H},\mathcal{S} \rangle\)}
        \begin{itemize}
            \item Soft Clause: clause that can be dropped, thus allowing $x_{i}$ to take any value
            \item Hard Clause: constrained clause used for encoding the representation of the predicted class
        \end{itemize}
    \end{block}
    \begin{block}{Goal}
        Find, gradually making the variables free, the minimum feature set $\mathcal{E} \subseteq \mathcal{S}$ while still keeping the pair $\langle \mathcal{H},\mathcal{S} \rangle$ unsatisfiable.
    \end{block}
\end{frame}

\begin{frame}{Abductive Explanations (AXps)}
    \begin{block}{Computing AXps}
        The set of features $\mathcal{X}$ is an AXp if the following formula is unsatisfiable:
        \begin{equation*}
            \left[ \bigwedge_{i \in \mathcal{X}} (x_i = v_i) \right] \land Enc(\tau(\mathbf{x}) \neq c)
        \end{equation*}
        This results in searching for a minimal subset $\mathcal{E} \subseteq \mathcal{S}$ such that
        \begin{equation*}
            \left[ \bigwedge_{(x_i = v_i) \in \mathcal{E}} (x_i = v_i) \right] \land Enc(\tau(\mathbf{x}) \neq c)
        \end{equation*} 
        is unsatisfiable.
        \vspace{2pt}
    \end{block}
\end{frame}

